
\documentclass[11pt]{article}
%%% style file you will need for some commands%%%%%%%%%%%%%%%%%%%%%%
%% aaHWbeginner is the style file I have used to typeset many commands, feel free to use them in your solutions.
%% bear in mind that if you need to define your command then you will have to make sure that it is not in conflict to my pre-defined command. Otherwise you will need to either use
%% commands defined by me or edit the style file appropriately.

%%\usepackage{anurag}
\usepackage{aaHWbeginner}
\usepackage{systeme} %% this package is to typeset system of equations.
%%\sysdelim..  %% this is to supress the delimiters from the system of equations when using the "systeme" package
\usepackage{tikz}
\usetikzlibrary{shapes,arrows,matrix,decorations.markings}
\newcommand*\circled[1]{\tikz[baseline=(char.base)]{
            \node[shape=circle,draw,inner sep=2pt] (char) {#1};}}
%%the \circled command has been used to create text inside circle for grading table.
\tikzstyle{blk}=[circle,inner sep=0pt,minimum size =2pt,draw,fill=black,line width=0.8pt]
\tikzstyle{blanknode}=[circle,inner sep=2pt,minimum size =8pt,draw,line width=0.8pt]
%%following is to have a directed edge with arrow anywhere in between, name of the style is ->- and it requres an input for the position of the arrow.
%%requires decorations.markings library
%% example usgae \draw[->-=0.5] (0,0) to [bend left] (2,4);
%% see http://tex.stackexchange.com/questions/39278/tikz-arrowheads-in-the-center
\tikzset{->-/.style={decoration={markings,mark=at position #1 with {\arrow{>}}},postaction={decorate}}}


%%%\geometry{letterpaper, textwidth=17cm, textheight=22cm}

%%%%%%%%%%%%%%%%%%%%%%%%%%%%%%%%%%%% THE FOLLOWING IS FOR THE COVER SHEET--FILL IN appropriately%%
\newcommand{\mycourse}{MATH-241}
\newcommand{\semesteryear}{Fall 2015}
\newcommand{\myname}{TYPE YOUR NAME here}  %%TYPE in YOUR NAME HERE  <<<<<<<<<<<<<<<|========================================= (PLEASE PUT YOUR NAME HERE)==========
\newcommand{\hwnumber}{3} %%TYPE in the HW  number 1,2,3,.. HERE  <<<<<<<<<<<<<<<|========================================= (PLEASE PUT the HW number here)==========
%%%%%%%%%%%%%%%%%%%%%%%%%%%%%%%%%%%%%% cover sheet preamble ends here

%%%%%%%%FOLLOWING counter IS TO AUTOMATE THE NUMBERING OF THE PROBLEMS%%%%%%%%%%%%%
\newcounter{Quesnumb}  %% this creates the counter
\setcounter{Quesnumb}{0} %% this sets a specific value to the counter
%%%%%% the following new command can be used to increment and print the counter  http://chenfuture.wordpress.com/2007/12/31/a-simple-counter/
\newcommand{\problemnum}{%
            \addtocounter{Quesnumb}{1}%
            \arabic{Quesnumb}}


%%%% following is NOT TO BE EDITED, DO NOT TYPE ANYTHING HERE, it will receive inputs from what you fill above%%%%%%%%%%%%%%%%%%
\title{\textbf{\mycourse} \hfill Homework \hwnumber \hfill \textbf{\semesteryear}} %% DO NOT type in HW number,  mycourse and semesteryear values
\author{\myname} %% DO NOT type in your name here.
\date{ \textbf{DUE DATE Sep 29, 2015 by 11:00PM (in \textsc{Dropbox})}}
%%%%%%%%%%%%%%%%%%%%%%%%%%%%%%%%%%%%%%%%%%%%%%%%%%%%%%%%%%%%%%%%%%%%%%%%%%%%%%%%%%%%%%%%%%%%%%%%%%%%%%%%%%%%%%%%%%%%%%%%%%%%%%%%

\setlength{\parindent}{0pt} %% paragraphs will not be indented
\setlength{\parskip}{.25cm} %% space between paragraphs
\linespread{1.1}

\begin{document}
\thispagestyle{empty} %%this is to supress the page number on the cover page
%% this file is for LaTeX users
%% the %% this file is for LaTeX users
%% the %% this file is for LaTeX users
%% the \include{hwcover} in the preamble of your HW file takes this file as an input and renders all the information appropriately.
%% this file should be in the same directory as your HW files.
%% DO NOT type your name etc. here in this file. See the preamble of your HW template file, where you need to write those inputs.
\newcommand{\circlescale}{& \circled{5} \hspace*{0.2cm} \circled{4} \hspace*{0.2cm} \circled{3} \hspace*{0.2cm} \circled{2} \hspace*{0.2cm} \circled{0}\\\midrule}
\begin{minipage}{6in}
    {\Large \mycourse \hfill Linear Algebra \hfill \semesteryear}
    \begin{center}
    %%\textsf{\huge Cover Sheet for HW}\\
    %%\vspace*{0.3cm}
    \fbox{\parbox[][1cm][c]{10cm}{\sc \Large Print Name: \myname}}
    \end{center}
\end{minipage}
\vspace*{0.3cm}

\textbf{Read the instructions below:}
\begin{itemize}
    %%\item \textsf{STAPLE} your homework.
    \item Solutions should be supported with reasons. Just writing the numerical answers is not enough and NO credit will be given for that.
    \item Please submit both the \LaTeX and PDF files. Refer to the instructions on the HW page about submitting the homework .
\end{itemize}

\begin{center}
\fbox{\parbox[][0.8cm][c]{10cm}{{\sc Assignment \# \hwnumber}}}
\end{center}

%%%for the \circled command used below see the file hw_template.tex. I have used TikZ to do this.

\renewcommand{\arraystretch}{1.3}
\begin{tabular}{lc}
  \toprule[2pt]
  % after \\: \midrule or \cline{col1-col2} \cline{col3-col4} ...
  \multicolumn{2}{c}{\bfseries For Grading Only}\\ \toprule[2pt]
  Completion Points \circlescale
  \#1               \circlescale
  \#2               \circlescale
  \#3               \circlescale
  \#4               \circlescale
  \#5               \circlescale
  \#6               \circlescale
  \#7               \circlescale
  \#8               \circlescale
  \#9               \circlescale
  \textbf{Total} & \\
  \bottomrule[2pt]
\end{tabular}
 in the preamble of your HW file takes this file as an input and renders all the information appropriately.
%% this file should be in the same directory as your HW files.
%% DO NOT type your name etc. here in this file. See the preamble of your HW template file, where you need to write those inputs.
\newcommand{\circlescale}{& \circled{5} \hspace*{0.2cm} \circled{4} \hspace*{0.2cm} \circled{3} \hspace*{0.2cm} \circled{2} \hspace*{0.2cm} \circled{0}\\\midrule}
\begin{minipage}{6in}
    {\Large \mycourse \hfill Linear Algebra \hfill \semesteryear}
    \begin{center}
    %%\textsf{\huge Cover Sheet for HW}\\
    %%\vspace*{0.3cm}
    \fbox{\parbox[][1cm][c]{10cm}{\sc \Large Print Name: \myname}}
    \end{center}
\end{minipage}
\vspace*{0.3cm}

\textbf{Read the instructions below:}
\begin{itemize}
    %%\item \textsf{STAPLE} your homework.
    \item Solutions should be supported with reasons. Just writing the numerical answers is not enough and NO credit will be given for that.
    \item Please submit both the \LaTeX and PDF files. Refer to the instructions on the HW page about submitting the homework .
\end{itemize}

\begin{center}
\fbox{\parbox[][0.8cm][c]{10cm}{{\sc Assignment \# \hwnumber}}}
\end{center}

%%%for the \circled command used below see the file hw_template.tex. I have used TikZ to do this.

\renewcommand{\arraystretch}{1.3}
\begin{tabular}{lc}
  \toprule[2pt]
  % after \\: \midrule or \cline{col1-col2} \cline{col3-col4} ...
  \multicolumn{2}{c}{\bfseries For Grading Only}\\ \toprule[2pt]
  Completion Points \circlescale
  \#1               \circlescale
  \#2               \circlescale
  \#3               \circlescale
  \#4               \circlescale
  \#5               \circlescale
  \#6               \circlescale
  \#7               \circlescale
  \#8               \circlescale
  \#9               \circlescale
  \textbf{Total} & \\
  \bottomrule[2pt]
\end{tabular}
 in the preamble of your HW file takes this file as an input and renders all the information appropriately.
%% this file should be in the same directory as your HW files.
%% DO NOT type your name etc. here in this file. See the preamble of your HW template file, where you need to write those inputs.
\newcommand{\circlescale}{& \circled{5} \hspace*{0.2cm} \circled{4} \hspace*{0.2cm} \circled{3} \hspace*{0.2cm} \circled{2} \hspace*{0.2cm} \circled{0}\\\midrule}
\begin{minipage}{6in}
    {\Large \mycourse \hfill Linear Algebra \hfill \semesteryear}
    \begin{center}
    %%\textsf{\huge Cover Sheet for HW}\\
    %%\vspace*{0.3cm}
    \fbox{\parbox[][1cm][c]{10cm}{\sc \Large Print Name: \myname}}
    \end{center}
\end{minipage}
\vspace*{0.3cm}

\textbf{Read the instructions below:}
\begin{itemize}
    %%\item \textsf{STAPLE} your homework.
    \item Solutions should be supported with reasons. Just writing the numerical answers is not enough and NO credit will be given for that.
    \item Please submit both the \LaTeX and PDF files. Refer to the instructions on the HW page about submitting the homework .
\end{itemize}

\begin{center}
\fbox{\parbox[][0.8cm][c]{10cm}{{\sc Assignment \# \hwnumber}}}
\end{center}

%%%for the \circled command used below see the file hw_template.tex. I have used TikZ to do this.

\renewcommand{\arraystretch}{1.3}
\begin{tabular}{lc}
  \toprule[2pt]
  % after \\: \midrule or \cline{col1-col2} \cline{col3-col4} ...
  \multicolumn{2}{c}{\bfseries For Grading Only}\\ \toprule[2pt]
  Completion Points \circlescale
  \#1               \circlescale
  \#2               \circlescale
  \#3               \circlescale
  \#4               \circlescale
  \#5               \circlescale
  \#6               \circlescale
  \#7               \circlescale
  \#8               \circlescale
  \#9               \circlescale
  \textbf{Total} & \\
  \bottomrule[2pt]
\end{tabular}
 %% make sure you have the file "hwcover.tex" in the same folder as your actual homework file
\renewcommand{\arraystretch}{1} %% this is to make sure that array stretch in "hwcover" is nuetralized.

\clearpage %% these are to reset the page number for the first page of your homework to 1.
\pagenumbering{arabic} %% these are to reset the page number for the first page of your homework to 1.
\maketitle

%%%%%%%%%%%%%%%%%%%%%%%%%%%%%%%%% YOU MAY START TYPING YOUR ANSWERS BELOW %%%%%%%%%%%%%%%%%%%%%%%%%%%%%%%%%%%%%%%
%%%%%%%%%%%%%%%%%%%%%%%%%%%%%%%%%%%%%%%%%%%%%%%%%%%%%%%%%%%%%%%%%%%%%%%%%%%%%%%%%%%%%%%%%%%%%%%%%%%%%%%%%%%%%%%%%
%% NOTE: In my style file aaHWbeginner.sty I have defined two environments "problem" and "solution" that can be used to type in your question and answer respectively as shown below.%%

%%{\red Use the ``problem'' environment for questions and the ``solution'' environment to type the solutions.}

\begin{problem}{\problemnum \, \textsf{(POOLE 2.3.12)}}
    Show that $\bbR^3$ = span$\parens{\vvv{ 1}{ 1}{-1}
                                      \vvv{ 1}{ 1}{ 1}
                                      \vvv{ 1}{-1}{ 1}}$.
\end{problem}
\begin{solution}

\end{solution}

\begin{problem}{\problemnum \, \textsf{(POOLE 2.3.16)}}
    Describe the set of given vectors (a) geometrically, and (b) algebraically.
    $$ \vvv{  }{  }{ }
       \vvv{  }{  }{ }
       \vvv{  }{  }{ }
    $$
\end{problem}
\begin{solution}
\end{solution}

\begin{problem}{\problemnum \, \textsf{(POOLE 1.2.48)}}
    Find all values of the scalar $k$ for which the two
    vectors are orthagonal
    \[ \bu = \vect{2}{3},
        \bv = \vect{k + 1}{k - 1}
    \]
\end{problem}
\begin{solution}\\
    In $\mathbb{R}^2$:
    \[ \frac{\bu \cdot \bv}
            {\norm*{\bu} \norm*{\bv}}
        = \cos(\pi/2)
        = 0
        \implies \bu \text{ is perpendicular to } \bv
    \]
    Reifiying, we have:
    \begin{align*}
        \frac{\vect{2}{3} \cdot \vect{k+1}{k-1}}
             {\norm*{\vect{2}{3}}\norm*{\vect{k+1}{k-1}}} &= 0
        \\\\
        \frac{2k + 2 + 3k - 3}
             {\sqrt{13}\sqrt{(k + 1)^{2} + (k - 1)^{2}}} &= 0
        \\\\
        \frac{5k - 1}
             { \sqrt{13}\sqrt{2k^{2} + 2}} &= 0
        \\\\
        \frac{5k - 1}
             {\sqrt{2}\sqrt{13}\sqrt{k^2 + 1}} &= 0
        \\\\
        \frac{5k - 1}
             {\sqrt{k^2 + 1}} &= 0
        \\\\
        \frac{5k}{\sqrt{k^2 + 1}}
        &=
        \frac{1}{\sqrt{k^2 + 1}}
        \\\\
        5k &= 1
        \\\\
        k &= \frac{1}{5}
    \end{align*}
\end{solution}

\begin{problem}{\problemnum \, \textsf{(POOLE 2.3.24)}}

\end{problem}


\begin{problem}{\problemnum \, \textsf{(POOLE 2.3.30)}}

\end{problem}

\begin{problem}{\problemnum \, \textsf{(POOLE 2.3.42)}}

\end{problem}

\begin{problem}{\problemnum} 
Let $\bu, \bv, \bw$ be vectors such that $\norm{\bu}=1, \norm{\bv}=2$ and $\norm{\bw}=3$, $\bu$ is orthogonal to $\bv$, and that the angle between $\bu$ and $\bw$ is $\pi/3$ and that between $\bv$ and $\bw$ is $\pi/6$. Find $\norm{\bu+\bv+\bw}$.
\end{problem}
\begin{solution}\\
    Solving for $\bu \cdot \bv$:
    \begin{align*}
        & \frac{\bu \cdot \bv}
               {\norm{\bu}\norm{\bv}} = 0 = \cos(\frac{\pi}{2})
        \\
        \implies& \bu \cdot \bv = 0
    \end{align*}
    Solving for $\bu \cdot \bw$:
    \begin{align*}
        & \frac{\bu \cdot \bw}
        {\norm{\bu}\norm{\bw}} = \frac{1}{2} = \cos(\frac{\pi}{3})
        \\
        \implies& \bu \cdot \bw = \frac{3}{2}
    \end{align*}
    Solving for $\bu \cdot \bw$:
    \begin{align*}
        & \frac{\bv \cdot \bw}
        {\norm{\bv}\norm{\bw}} = \frac{\sqrt{3}}{2} = \cos(\frac{\pi}{6})
        \\
        \implies& \bv \cdot \bw = 3\sqrt{3}
    \end{align*}
    Rewriting $\norm{\bu+\bv+\bw}$
    \begin{align*}
        & \norm{\bu+\bv+\bw}
        \\
        =& \sqrt{(\bu+\bv+\bw) \cdot (\bu+\bv+\bw)}
        \\
        =& \sqrt{\bu \cdot (\bu+\bv+\bw)
                      + \bv \cdot (\bu+\bv+\bw)
                      + \bw \cdot (\bu+\bv+\bw)}
        \\
        =& \sqrt{\bu \cdot \bu
                      + \bv \cdot \bv
                      + \bw \cdot \bw
                      + 2(\bu \cdot \bv)
                      + 2(\bu \cdot \bw)
                      + 2(\bv \cdot \bw)}
        \\
        =& \sqrt{\norm{\bu}^{2}
                      + \norm{\bv}^{2}
                      + \norm{\bw}^{2}
                      + 2(\bu \cdot \bv)
                      + 2(\bu \cdot \bw)
                      + 2(\bv \cdot \bw)}
        \\
        =& \sqrt{1^2 + 2^2 + 3^2 + 2(0) + 2(3/2) + 2(3\sqrt{3})}
        \\
        =& \sqrt{17+6\sqrt{3}}
    \end{align*}

\end{solution}

\begin{problem}{\problemnum}
Describe the possible row reduced echelon forms (rref) of the matrix $\bA$ for each of the following:
\begin{enumerate}
    \item $\bA$ is a $3 \times 3$ matrix with linearly independent columns.
    \item $\bA = [\bv_1 \quad \bv_2 \quad \bv_3]$ is a $4 \times 3$ matrix, such that $\{\bv_1, \bv_2\}$ is linearly independent and $\bv_3 \notin \text{Span}(\bv_1,\bv_2)$.
\end{enumerate}
\end{problem}

\begin{problem}{\problemnum}
Let $\bA$ be a $7 \times 4$ matrix with linearly independent columns and $\bR$ be the row reduced echelon form of $\bA$.
    \begin{enumerate}
        \item Find the number of zero rows of $\bR$. Support your answer with reasons.
        \item Based on the information given complete the following sentence: \textsf{the columns of $\bA$ will span a \makebox[0.5in]{\hrulefill} dimensional space in a \makebox[0.5in]{\hrulefill} dimensional ambient space.}
        \item What can be said about the linear span of the rows of $\bA$ (be careful: rows of $\bA$ live in a different world than the columns)?
        \item Explain why $\bA^{T}\by = \begin{pmatrix}0&1&0&1\end{pmatrix}^{T}$ is a consistent system.
    \end{enumerate}
\end{problem}

\begin{problem}{\problemnum} 
Let $L$ be the line that passes through the points $\begin{pmatrix}-2\\1\\1\\0\end{pmatrix}$ and $\begin{pmatrix}5\\10\\-1\\4\end{pmatrix}$.
    \begin{enumerate}
        \item Find the vector equation of line $L$.
        \item Does the point $\begin{pmatrix}1\\0\\2\\1\end{pmatrix}$ lie on the line. Give reasons.
        \item Find the equation of a line that passes through the origin and is orthogonal to $L$. How many such distinct lines are possible?
    \end{enumerate}
\end{problem}
\begin{solution}
    \begin{enumerate}
        \item
            \begin{align*}
                A = \vvvv{-2}{1}{1}{0},
                B = \vvvv{5}{10}{-1}{4},
                \bv = B - A = \vvvv{7}{9}{-2}{4}
                \\\\
                L = \vvvv{-2}{1}{1}{0}
                  + t\vvvv{7}{9}{-2}{4},
                  \forall t \in \bbR
            \end{align*}
        \item
            \begin{proof}\begin{align*}
                \vvvv{1}{0}{2}{1} = \vvvv{-2 + 7t}{1 + 9t}{1 - 2t}{4t}
                \\
                \implies \frac{3}{7} = t = \frac{-1}{9}
                \\
                \text{Contradiction.}
                \\
                \vvvv{1}{0}{2}{1} \notin L
            \end{align*}\end{proof}
        \item
            \begin{align*}
                \text{Let $\bu$ be the direction vector in:}\\
                M &= \vvvv{0}{0}{0}{0} + s\vvvv{x}{y}{z}{w}
                \\
                \bu \cdot \bv = 0 \implies 7\bx + 9\by -2\bz + 4\bw
                \\
                \text{One solution: }
                \\
                \vvvv{x}{y}{z}{w} = \vvvv{1}{-1}{1}{1}
            \end{align*}
            Only one unique solution exists.
    \end{enumerate}
\end{solution}

\begin{problem}{\problemnum}
In $\bbR^4$ let
\[
A = \text{Span }\left\{\begin{bmatrix}[r]1\\2\\0\\-1\end{bmatrix}, \begin{bmatrix}[r]-1\\1\\1\\1\end{bmatrix}\right\} \qquad B = \text{Span }\left\{\begin{bmatrix}0\\0\\1\\1\end{bmatrix}, \begin{bmatrix}2\\2\\2\\2\end{bmatrix}\right\}
\]
Describe the set $A \cap B$?

Give both: the geometrical interpretation (it is a plane, line etc..) as well as the algebraic description. By an algebraic description we mean something that we do when writing the solution set of a system of equations.
\end{problem}

\begin{problem}{\problemnum}
Let $S=\{\bv_1, \bv_2, \bv_3\}$ be a linearly independent set of vectors. For what value of $k$ will the set $T=\{\bv_2-\bv_1, k\bv_3-\bv_2, \bv_1-\bv_3\}$ be linearly independent.
\end{problem}

\begin{problem}{\problemnum}
Let $S=\{\bv_1, \bv_2, \bv_3, \bv_4\}$ be a set of four non-zero vectors in $\bbR^{10}$ such that they are pairwise orthogonal to each other (i.e. $\bv_i \cdot \bv_j=0$ for $i \neq j$). Prove that the set $S$ is a linearly independent set. You should start with $x_1\bv_1+ \dotsb + x_4\bv_4=\mathbf{0}$ and then prove that the ONLY solution for this is $x_1=\ldots=x_4=0$ (the zero solution).
\end{problem}


\end{document}

%%%\item \textbf{Section 3.1 : } 24, 30, 36
%%\begin{center}
%%{\sc Following problems are for \textbf{\blue PRACTICE ONLY}. No need to turn these in.}
%%\end{center}
%%\item Let $\bx$ be a vector such that $\bx=\bu+5\bv+12\bw$, where $\bu, \bv$ and $\bw$ are vectors in $\bbR^n$. Show that $\bx \in \text{Span}(\bu+\bv, \bv+\bw, \bw+\bu)$.
%%
%%\item
%%\textbf{Linear independence and dependence} are properties of sets. So
%%one can ask questions related to set theoretic operations. If the following statements are true then give a proof of it, otherwise give a counterexample with complete explanation of how it violates the statement.\\
%% If $A$ and $B$ are two sets of linearly independent vectors in
%% $\bbR^n$ then
%% \begin{enumerate}
%%    \item Can $A \cap B$ be linearly independent? Must it be.
%%    \item Can $A \cup B$ be linearly independent? Must it be.
%% \end{enumerate}
%%
%%\item Suppose $\bv_1, \bv_2, \bv_3, \bv_4$ are vectors in $\bbR^n$ such that $\bv_1+2\bv_2+\bv_3+\bv_4=\mathbf{0}$.
%%\begin{enumerate}
%%    \item If $\bw = c_1\bv_1 + c_2\bv_2 +c_3\bv_3 + c_4\bv_4$, then express $\bw$ as a linear combination of $\bv_1, \bv_2, \bv_3$ only.
%%    \item Show that span($\bv_1,\bv_2, \bv_3) = \mbox{span}(\bv_1, \bv_2, \bv_3, \bv_4)$.
%%\end{enumerate}
%%
%%\item
%%\begin{enumerate}
%%    \item Show that any set of $4$ vectors in $\bbR^2$ is always linearly dependent.
%%    \item Does the same thing hold for a set of $5$ vectors? What about $3$ vectors? Do you see a general phenomenon related to the number
%%        of solutions of a homogeneous system.
%%    \item What is the most number of elements a linearly independent subset of $\bbR^2$ can have?
%%\end{enumerate}
%%
%%\item Determine the value(s) of $a$ such that $\left\{\begin{pmatrix} 1\\ a\end{pmatrix}, \begin{pmatrix} a\\ a+2\end{pmatrix}\right\}$ is a linearly independent set.
%%
%%\item How many pivot columns must a $7 \times 5$ matrix have if its columns are linearly independent? Why? Will your answer change if it was $5 \times 7$ matrix? Why?
%%\end{enumerate}
%%

%\begin{problem}{\problemnum}
%Find the largest possible number of independent vectors among
%\[\bv_1=
%\begin{pmatrix}
%1\\-1\\0\\0
%\end{pmatrix}
%\,
%\bv_2=
%\begin{pmatrix}
%1\\0\\-1\\0
%\end{pmatrix}
%\,
%\bv_3=
%\begin{pmatrix}
%1\\0\\0\\-1
%\end{pmatrix}
%\,
%\bv_4=
%\begin{pmatrix}
%0\\1\\-1\\0
%\end{pmatrix}
%\,
%\bv_5=
%\begin{pmatrix}
%0\\1\\0\\-1
%\end{pmatrix}
%\,
%\bv_6=
%\begin{pmatrix}
%0\\0\\1\\-1
%\end{pmatrix}
%\]
%\end{problem}
