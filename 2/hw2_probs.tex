\documentclass[11pt]{article}
%%% style file you will need for some commands%%%%%%%%%%%%%%%%%%%%%%
%% aaexam is the style file I have used to typeset many commands, feel free to use them in your solutions.
%% bear in mind that if you need to define your command then you will have to make sure that it is not in conflict to my pre-defined command. Otherwise you will need to either use
%% commands defined by me or edit the style file appropriately.

%%\usepackage{anurag}
\usepackage{aaHWbeginner}
\usepackage{systeme} %% this package is to typeset system of equations.
%%\sysdelim..  %% this is to supress the delimiters from the system of equations when using the "systeme" package
\usepackage{tikz}
\usetikzlibrary{shapes,arrows,matrix,decorations.markings}
\newcommand*\circled[1]{\tikz[baseline=(char.base)]{
            \node[shape=circle,draw,inner sep=2pt] (char) {#1};}}
%%the \circled command has been used to create text inside circle for grading table.
\tikzstyle{blk}=[circle,inner sep=0pt,minimum size =2pt,draw,fill=black,line width=0.8pt]
\tikzstyle{blanknode}=[circle,inner sep=2pt,minimum size =8pt,draw,line width=0.8pt]
%%following is to have a directed edge with arrow anywhere in between, name of the style is ->- and it requres an input for the position of the arrow.
%%requires decorations.markings library
%% example usgae \draw[->-=0.5] (0,0) to [bend left] (2,4);
%% see http://tex.stackexchange.com/questions/39278/tikz-arrowheads-in-the-center
\tikzset{->-/.style={decoration={markings,mark=at position #1 with {\arrow{>}}},postaction={decorate}}}


%%%\geometry{letterpaper, textwidth=17cm, textheight=22cm}

%%%%%%%%%%%%%%%%%%%%%%%%%%%%%%%%%%%% THE FOLLOWING IS FOR THE COVER SHEET--FILL IN appropriately%%
\newcommand{\mycourse}{MATH-241}
\newcommand{\semesteryear}{Fall 2015}
\newcommand{\myname}{TYPE YOUR NAME here}  %%TYPE in YOUR NAME HERE  <<<<<<<<<<<<<<<|========================================= (PLEASE PUT YOUR NAME HERE)==========
\newcommand{\hwnumber}{2} %%TYPE in the HW  number 1,2,3,.. HERE  <<<<<<<<<<<<<<<|========================================= (PLEASE PUT the HW number here)==========
%%%%%%%%%%%%%%%%%%%%%%%%%%%%%%%%%%%%%% cover sheet preamble ends here

%%%%%%%%FOLLOWING counter IS TO AUTOMATE THE NUMBERING OF THE PROBLEMS%%%%%%%%%%%%%
\newcounter{Quesnumb}  %% this creates the counter
\setcounter{Quesnumb}{0} %% this sets a specific value to the counter
%%%%%% the following new command can be used to increment and print the counter  http://chenfuture.wordpress.com/2007/12/31/a-simple-counter/
\newcommand{\problemnum}{%
            \addtocounter{Quesnumb}{1}%
            \arabic{Quesnumb}}


%%%% following is NOT TO BE EDITED, DO NOT TYPE ANYTHING HERE, it will receive inputs from what you fill above%%%%%%%%%%%%%%%%%%
\title{\textbf{\mycourse} \hfill Homework \hwnumber \hfill \textbf{\semesteryear}} %% DO NOT type in HW number,  mycourse and semesteryear values
\author{\myname} %% DO NOT type in your name here.
\date{ \textbf{DUE DATE Sep 22, 2015 by 11:00PM (in \textsc{Dropbox})}}} 
%%%%%%%%%%%%%%%%%%%%%%%%%%%%%%%%%%%%%%%%%%%%%%%%%%%%%%%%%%%%%%%%%%%%%%%%%%%%%%%%%%%%%%%%%%%%%%%%%%%%%%%%%%%%%%%%%%%%%%%%%%%%%%%%

\setlength{\parindent}{0pt} %% paragraphs will not be indented
\setlength{\parskip}{.25cm} %% space between paragraphs
\linespread{1.1}

\begin{document}
\thispagestyle{empty} %%this is to supress the page number on the cover page
%% this file is for LaTeX users
%% the %% this file is for LaTeX users
%% the %% this file is for LaTeX users
%% the \include{hwcover} in the preamble of your HW file takes this file as an input and renders all the information appropriately.
%% this file should be in the same directory as your HW files.
%% DO NOT type your name etc. here in this file. See the preamble of your HW template file, where you need to write those inputs.
\newcommand{\circlescale}{& \circled{5} \hspace*{0.2cm} \circled{4} \hspace*{0.2cm} \circled{3} \hspace*{0.2cm} \circled{2} \hspace*{0.2cm} \circled{0}\\\midrule}
\begin{minipage}{6in}
    {\Large \mycourse \hfill Linear Algebra \hfill \semesteryear}
    \begin{center}
    %%\textsf{\huge Cover Sheet for HW}\\
    %%\vspace*{0.3cm}
    \fbox{\parbox[][1cm][c]{10cm}{\sc \Large Print Name: \myname}}
    \end{center}
\end{minipage}
\vspace*{0.3cm}

\textbf{Read the instructions below:}
\begin{itemize}
    %%\item \textsf{STAPLE} your homework.
    \item Solutions should be supported with reasons. Just writing the numerical answers is not enough and NO credit will be given for that.
    \item Please submit both the \LaTeX and PDF files. Refer to the instructions on the HW page about submitting the homework .
\end{itemize}

\begin{center}
\fbox{\parbox[][0.8cm][c]{10cm}{{\sc Assignment \# \hwnumber}}}
\end{center}

%%%for the \circled command used below see the file hw_template.tex. I have used TikZ to do this.

\renewcommand{\arraystretch}{1.3}
\begin{tabular}{lc}
  \toprule[2pt]
  % after \\: \midrule or \cline{col1-col2} \cline{col3-col4} ...
  \multicolumn{2}{c}{\bfseries For Grading Only}\\ \toprule[2pt]
  Completion Points \circlescale
  \#1               \circlescale
  \#2               \circlescale
  \#3               \circlescale
  \#4               \circlescale
  \#5               \circlescale
  \#6               \circlescale
  \#7               \circlescale
  \#8               \circlescale
  \#9               \circlescale
  \textbf{Total} & \\
  \bottomrule[2pt]
\end{tabular}
 in the preamble of your HW file takes this file as an input and renders all the information appropriately.
%% this file should be in the same directory as your HW files.
%% DO NOT type your name etc. here in this file. See the preamble of your HW template file, where you need to write those inputs.
\newcommand{\circlescale}{& \circled{5} \hspace*{0.2cm} \circled{4} \hspace*{0.2cm} \circled{3} \hspace*{0.2cm} \circled{2} \hspace*{0.2cm} \circled{0}\\\midrule}
\begin{minipage}{6in}
    {\Large \mycourse \hfill Linear Algebra \hfill \semesteryear}
    \begin{center}
    %%\textsf{\huge Cover Sheet for HW}\\
    %%\vspace*{0.3cm}
    \fbox{\parbox[][1cm][c]{10cm}{\sc \Large Print Name: \myname}}
    \end{center}
\end{minipage}
\vspace*{0.3cm}

\textbf{Read the instructions below:}
\begin{itemize}
    %%\item \textsf{STAPLE} your homework.
    \item Solutions should be supported with reasons. Just writing the numerical answers is not enough and NO credit will be given for that.
    \item Please submit both the \LaTeX and PDF files. Refer to the instructions on the HW page about submitting the homework .
\end{itemize}

\begin{center}
\fbox{\parbox[][0.8cm][c]{10cm}{{\sc Assignment \# \hwnumber}}}
\end{center}

%%%for the \circled command used below see the file hw_template.tex. I have used TikZ to do this.

\renewcommand{\arraystretch}{1.3}
\begin{tabular}{lc}
  \toprule[2pt]
  % after \\: \midrule or \cline{col1-col2} \cline{col3-col4} ...
  \multicolumn{2}{c}{\bfseries For Grading Only}\\ \toprule[2pt]
  Completion Points \circlescale
  \#1               \circlescale
  \#2               \circlescale
  \#3               \circlescale
  \#4               \circlescale
  \#5               \circlescale
  \#6               \circlescale
  \#7               \circlescale
  \#8               \circlescale
  \#9               \circlescale
  \textbf{Total} & \\
  \bottomrule[2pt]
\end{tabular}
 in the preamble of your HW file takes this file as an input and renders all the information appropriately.
%% this file should be in the same directory as your HW files.
%% DO NOT type your name etc. here in this file. See the preamble of your HW template file, where you need to write those inputs.
\newcommand{\circlescale}{& \circled{5} \hspace*{0.2cm} \circled{4} \hspace*{0.2cm} \circled{3} \hspace*{0.2cm} \circled{2} \hspace*{0.2cm} \circled{0}\\\midrule}
\begin{minipage}{6in}
    {\Large \mycourse \hfill Linear Algebra \hfill \semesteryear}
    \begin{center}
    %%\textsf{\huge Cover Sheet for HW}\\
    %%\vspace*{0.3cm}
    \fbox{\parbox[][1cm][c]{10cm}{\sc \Large Print Name: \myname}}
    \end{center}
\end{minipage}
\vspace*{0.3cm}

\textbf{Read the instructions below:}
\begin{itemize}
    %%\item \textsf{STAPLE} your homework.
    \item Solutions should be supported with reasons. Just writing the numerical answers is not enough and NO credit will be given for that.
    \item Please submit both the \LaTeX and PDF files. Refer to the instructions on the HW page about submitting the homework .
\end{itemize}

\begin{center}
\fbox{\parbox[][0.8cm][c]{10cm}{{\sc Assignment \# \hwnumber}}}
\end{center}

%%%for the \circled command used below see the file hw_template.tex. I have used TikZ to do this.

\renewcommand{\arraystretch}{1.3}
\begin{tabular}{lc}
  \toprule[2pt]
  % after \\: \midrule or \cline{col1-col2} \cline{col3-col4} ...
  \multicolumn{2}{c}{\bfseries For Grading Only}\\ \toprule[2pt]
  Completion Points \circlescale
  \#1               \circlescale
  \#2               \circlescale
  \#3               \circlescale
  \#4               \circlescale
  \#5               \circlescale
  \#6               \circlescale
  \#7               \circlescale
  \#8               \circlescale
  \#9               \circlescale
  \textbf{Total} & \\
  \bottomrule[2pt]
\end{tabular}
 %% make sure you have the file "hwcover.tex" in the same folder as your actual homework file
\renewcommand{\arraystretch}{1} %% this is to make sure that array stretch in "hwcover" is nuetralized.

\clearpage %% these are to reset the page number for the first page of your homework to 1.
\pagenumbering{arabic} %% these are to reset the page number for the first page of your homework to 1.
\maketitle

%%%%%%%%%%%%%%%%%%%%%%%%%%%%%%%%% YOU MAY START TYPING YOUR ANSWERS BELOW %%%%%%%%%%%%%%%%%%%%%%%%%%%%%%%%%%%%%%%
%%%%%%%%%%%%%%%%%%%%%%%%%%%%%%%%%%%%%%%%%%%%%%%%%%%%%%%%%%%%%%%%%%%%%%%%%%%%%%%%%%%%%%%%%%%%%%%%%%%%%%%%%%%%%%%%%
%% NOTE: In my style file aaHWbeginner.sty I have defined two environments "problem" and "solution" that can be used to type in your question and answer respectively as shown below.%%

%%{\red Use the ``problem'' environment for questions and the ``solution'' environment to type the solutions.}

\begin{problem}{\problemnum \, \textsf{(POOLE 2.2.14)}}

\end{problem}

\begin{problem}{\problemnum \, \textsf{(POOLE 2.2.18)}}

\end{problem}

\begin{problem}{\problemnum \, \textsf{(POOLE 2.2.28)}}

\end{problem}

\begin{problem}{\problemnum \, \textsf{(POOLE 2.2.36)}}

\end{problem}

\begin{problem}{\problemnum \, \textsf{(POOLE 2.2.48)}}

\end{problem}

\begin{problem}{\problemnum}
In class we learnt how to use projections to find the distance of a point to a line or a point to a plane (you may also want to see \textbf{example 1.32 and 1.33}) in your textbook. Here we want to use similar ideas by finding the (perpendicular) distance between two lines.

Suppose we are given two lines $L_1$ and $L_2$ as follows:
\[\br_1=\begin{bmatrix}1\\0\\-2 \end{bmatrix}+s\begin{bmatrix}4\\1\\3 \end{bmatrix} \qquad \text{ and } \qquad \br_2=\begin{bmatrix}2\\-1\\0\end{bmatrix}+t\begin{bmatrix}2\\1\\0\end{bmatrix}.\]
To find the distance between these two lines we will follow the following recipe:
\begin{enumerate}
\item First find a vector $\mathbf{n}=(a,b,c)$ that is orthogonal to the direction vectors of both the lines. 
\item Then choose an arbitrary point $P$ on $L_1$ and an arbitrary point $Q$ on $L_2$. Find the vector $\overrightarrow{PQ}$.
\item Now use appropriate projections to find the distance between the two lines. 
\end{enumerate} 
\textsf{\red NOTE:} This technique cannot work in higher dimensions ($\geq 4$) because (as we have seen previously) for dimensions $\geq 4$, the vector $\mathbf{n}$ need not be unique.
\end{problem}

\begin{problem}{\problemnum}
Consider the system of equations:
\systeme[xyzw]{-w+2y+x=a, 2x+z= b, x+6y-z-3w=c} %%% systeme will specify variables alphabetically but by putting the optional [xyzw] we can specify the order of variables
\begin{enumerate}
	\item The \textsf{row picture} of this system can be described as the intersection of \textbf{three 3D hyperplanes} in the 4D space. How will you describe the \textsf{column picture} of this system?
	\item Find the condition on the parameters $a, b \textrm{ and } c$ so that the system is consistent.
	\item Does the system have a solution when $a=3, b=1$ and $c=-1$? If yes, then find the solution in vector form.
	\item Does the system have a solution when $a=0, b=1$ and $c=-1$? If yes, then find the solution in vector form.
\end{enumerate}
\end{problem}

\begin{problem}{\problemnum}
Consider the following augmented matrix of a system of equations:
\[
 \left[
     \begin{array}{rrr|r}
       1 & 2 & -3 &     4\\
       3 & -1 & 5 &     2\\
       4 & 1 & a^2-14 & a+2
     \end{array}
   \right]
\]
\begin{enumerate}
	\item Find the values of $a$ for which the matrix will have no pivots, one pivot, two pivots and three pivots.
	\item Based on your answers to the previous part, find the values of $a$ for which the system has NO solutions?
\textbf{exactly one} solution? \textbf{infinitely} many solutions?
\end{enumerate}
\end{problem}

\begin{problem}{\problemnum}
Determine the value(s) of $x$ for which the rank of the following matrix is \textbf{minimum}. 
\[
A = \begin{bmatrix}
         3&1&1&4\\
         x&4&10&1\\
         1&7&17&3\\
         2&2&4&3
    \end{bmatrix}
\]
\end{problem}

\begin{problem}{\problemnum}
Consider the following augmented matrix of a system of equations:
\[[\bA | \bb]=
 \left[
     \begin{array}{rrr|r}
       1 & 2 & 1 & 3\\
       0 & x & 5 & 10\\
       2 & 7 & x & y
     \end{array}
   \right]
\]
\begin{enumerate}
	\item Find all the values of $x$ and $y$ for which the \textbf{augmented matrix} $[\bA | \bb]$ has exactly two pivots.
	\item Find the values of $x$ for which the system has a unique solution.
    \item Find those pairs $(x,y)$ for which the system has more than one solution.
\end{enumerate}
\end{problem}

\begin{problem}{\problemnum} 
Answer the following:
	\begin{enumerate}
	\item If possible, give an example of a $3 \times 4$ matrix $\bA$ and $3 \times 1$ columns $\bb$ and $\bc$ such that $[\bA | \bb]$ is an augmented matrix for an inconsistent system but $[\bA | \bc]$ is an augmented matrix for a consistent system.
	\item Suppose a system of linear equation has $3 \times 5$ \textsf{augmented matrix} whose fifth column is a pivot column. Is the system consistent? (give reasons for your answer).
	\item Suppose a system of linear equation has $3 \times 5$ \textsf{coefficient matrix} with three pivot columns. Is the system consistent? (give reasons for your answer).
	\end{enumerate}
\end{problem}
	
\begin{problem}{\problemnum}
    \begin{enumerate}
    \item Suppose $\bA$ is a matrix such that $[\bA|\bu]$ and $[\bA|\bv]$ are both CONSISTENT systems. Show that $[\bA|(\bu+\bv)]$ is also a consistent system. \textit{(Hint: think in terms of column picture.)}
    \item Let \[\bu =
          \begin{bmatrix}
            0 \\
            2 \\
            4 \\
            6 \\
          \end{bmatrix}
        \qquad
    \bv =        \begin{bmatrix}
          1 \\
          3 \\
          5 \\
          7 \\
        \end{bmatrix}
    \]
Suppose $\bA$ is a matrix such that $[\bA|\bu]$ and $[\bA|\bv]$ are both INCONSISTENT systems. Is it possible that $[\bA|(\bu+\bv)]$ is a consistent system with infinitely many solutions.
If ``YES'', then give an example of such a matrix $\bA$ and show that it satisfies all the conditions given. If ``NO'', then give a reason why would such a matrix $\bA$ not exist.
\end{enumerate}
\end{problem}
\end{document}























%\begin{problem}{\problemnum}
%If the matrix
%\[
%A = \begin{bmatrix}
%         a & 1 & a & 0 & 0 & 0\\
%         0 & b & 1 & b & 0 & 0\\
%         0 & 0 & c & 1 & c & 0\\
%         0 & 0 & 0 & d & 1 & d
%    \end{bmatrix}
%\]
%has rank $r$, prove that:
%\begin{enumalph}
%    \item For all values of $a,b,c$, the rank $r \geq 3$.
%    \item $r=3$ if and only if $a=d=0$ and $bc=1$. \textsf{NOTE: ``if and only if'' means the statement is true both ways and hence you should prove both the statement and its converse.}
%    \item $r=4$ in all other cases.
%\end{enumalph}
%\end{problem}

%\begin{center}
%{\sc Following problems are for \textbf{\blue PRACTICE ONLY}. No need to turn these in.}
%\end{center}
%
%\item Describe the intersection of the three planes $u+v+w+z=6$ and $u+w+z=4$ and $u+w=2$ (all in 4D). Is it a line or a point or an empty set? What is the intersection if the fourth plane $u=-1$ is included? Find a fourth equation that leaves us with no solution.
%
%\item \textbf{Section 1.3 : } 7, 13, 19, 23
%
%\item \textbf{Section 2.1 : } 23, 36
%
%\item \textbf{Section 2.2 : } 13, 15, 17, 29, 33, 37, 43, 49, 55
%
%\item Prove that the following two matrices are NOT row equivalent\\
%\[
%\left[
%  \begin{array}{rrr}
%    2 & 0 & 0 \\
%    a & -1 & 0\\
%    b & c & 3
%  \end{array}
%\right]
%\qquad
%\left[
%  \begin{array}{rrr}
%    1 & 1 & 2 \\
%    -2 & 0 & -1\\
%    1 & 3 & 5
%  \end{array}
%\right]
%\]
%
%\item Forward elimination changes $\bA\bx = \bb$ to a row reduced $\mathbf{R}\bx = \mathbf{d}$: the complete solution is
%\[
%\bx=\begin{bmatrix}4\\0\\0\end{bmatrix}+c_1\begin{bmatrix}2\\1\\0\end{bmatrix}+c_2\begin{bmatrix}5\\0\\1\end{bmatrix}
%\]
%\begin{enumerate}
%	\item What is the $3 \times 3$ reduced row echelon matrix $\mathbf{R}$ and what is $\mathbf{d}$?
%	\item If the process of elimination subtracted 3 times row 1 from row 2 and
%		then 5 times row 1 from row 3, what matrix connects $\mathbf{R}$ and $\mathbf{d}$ to the original $\bA$ and
%		$\bb$? Use this matrix to find $\bA$ and $\bb$.
%\end{enumerate}
